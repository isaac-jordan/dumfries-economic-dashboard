% This example An LaTeX document showing how to use the l3proj class to
% write your report. Use pdflatex and bibtex to process the file, creating
% a PDF file as output (there is no need to use dvips when using pdflatex).

% Modified

\documentclass{l3proj}

\usepackage{wrapfig}

\begin{document}

\title{Team Project(H) Dissertation: Team A}

\author{Lewis Dicks - 2085749 \\
        Isaac Jordan - 2080466 \\
        Praxitelis (Branko) Kourtellos - 2060408 \\
        Christos (Takis) Nicolaides - 2084564  \\
        Rostislav Yordanov - 2074214 \\
        Michael Byars - 2028262}

\date{\today}

\maketitle

\begin{abstract}

For the first time at the University of Glasgow, the team project undertaken in the third year of a Computing Science
degree involves a real life customer, where previously students undertook projects for individuals lecturers in the
department. This was also the first year in which the Team Project was combined with the Professional Software
Development course as previously students were involved in two team projects, this is now one team with one project that
incorporates techniques learned from the Professional Software Development course. Students worked with institutions
based in the Dumfries and Galloway area of Scotland, such as the Crichton Regional Observatory, the NHS and the third
sector. For our team's project, the team were asked to create an interactive dashboard of various economic indicators that
would be integrated into the Crichton's homepage. This paper presents a case study of the nature of our project and
reflects on our experiences relating to software engineering practices and principles.

\end{abstract}

%% Comment out this line if you do not wish to give consent for your
%% work to be distributed in electronic format.
%% \educationalconsent

\newpage

%==============================================================================
\section{Introduction}
\label{sec:introduction}

%% Purpose of the document
This project report covers a 7 month University of Glasgow third year project. It is a case-study reporting on the aims of the project, the software development process adopted throughout development, the evolution of the software produced, and the evaluation and reflection of the group and project.

%% Very brief outline of the project
The team was asked to create an interactive dashboard of various economic indicators of the Dumfries and Galloway region. The idea behind this project was to provide an easy-to-access and instant visualisation of key economic data that allows businesses, individuals, and prospective investors to gain a snapshot of the current state of the regional economy without having to trawl through lots of data links.

Several key requirements of the final product that were identified included minimising clicks to get to an overview of information, having more detailed information available for those who needed it, and the dashboard should be viable to be managed by non-technical administration personnel.

The dashboard was built using the Python-based Django~\cite{Django} framework as the team had previous university experience building web applications using this so it was relatively straightforward to begin developing the initial components of the back-end of the website. In addition to this the team used many JavaScript-based front end technologies such as Angular JS~\cite{AngularWebpage}, d3~\cite{d3Webpage} and Gridster~\cite{AngularGridster}.

%% Summary of the rest of the document
The following pages attempt to comprehensively cover all aspects of the development of the dashboard including technical and non-technical issues. Section \ref{sec:background} presents background information on the project, and Section \ref{sec:team-dynamics} to Section \ref{sec:customer-management} cover a range of topics in detail.

%==============================================================================
\section{Case Study Background}
\label{sec:background}

For the first time the Level 3 Team Projects have real life customers, with real problems and requirements. In this first year of the scheme the Crichton Institute~\cite{CrichtonInstitute} has played a major role by offering 6 separate projects for teams to work on, including the dashboard project our team took on.

% Background on the Crichton Institute, Regional Observatory, and our contacts

The Crichton Institute was launched by the Cabinet Secretary for Education and Lifelong Learning, Michael Russel MSP in January 2013~\cite{CrichtonInsituteFounded}, with the aim of bringing together academic and business parters, local government, and regional development bodies in a joint effort in several key areas including: ``the rural economy and skills development, community development and placemaking, the impacts of demographic change and influencing policy and practice''~\cite{CrichtonInsituteCoreAims}. The institute aims to drive overall development of the region and improve it's ability to attract resources and inward investment~\cite{CrichtonInstituteAboutUs}.

In June 2014 the Crichton Institute launched a Regional Observatory~\cite{CrichtonInstituteRegionalObservatory}. The Regional Observatory is a ``web-based information and knowledge portal that acts as a one-stop open-access service for public data, information and intelligence about a wide range of social, economic, and environmental factors across Dumfries and Galloway and the South of Scotland''~\cite{ScotGovOpenDataResourcePack}. This aims to improve the resources of rural areas such as Dumfries and Galloway in terms of data gathering, access, and usage. The Regional Observatory also aims to provide a collaborative base of data and information from a variety of sources which is accessible to all~\cite{ScotGovOpenDataResourcePack}.

The client contacts the team interacted with were Development Officer Eva Milroy~\cite{EvaMilroyLinkedIn}, and Director Tony Fitzpatrick~\cite{TonyFitzpatrickLinkedIn}. Eva Milroy is a key member in setting up and continued running of the Crichton Institute Regional Observatory, as well as providing business intelligence, and developing employer engagement with the Crichton Institute. Tony Fitzpatrick has been director of the Crichton Institute Regional Observatory since it's inception. These two representatives were the team's main contact within the Crichton Institute, and they provided invaluable feedback and insight into the project and problem domain.

% Their problem description

The Regional Observatory provides open access to a large amount of data and reports on a variety of key economic indicators including population, housing, education, and crime. This information is summarised on the Regional Observatory site in text form~\cite{PopulationHealthReport}, however, there are no consistent visualisations of the data, nor the ability to get an overview of the region as a whole without consulting a considerable amount of specific data. This is the problem domain that the team took on to solve.

The team was presented with the idea of creating a regional dashboard which would provide quick access to an intuitive overview of the socioeconomic situation of the Dumfries and Galloway region. This dashboard would be publicly available to any and all interested parties looking for information on the region. The dashboard would contain visualisations of data relevant to Dumfries and Galloway, and comparisons to significant regions such as Scotland-wide. It would also contain links to the relevant sources of the data, so that users can ``drill down'' into the data they are most interested in.

The project also consisted of implementing a ``cheat sheet'' system. The customers wanted their users to be able to create a summary of the economic indicators most important to that specific user, and download it as a PDF document.

The customers provided examples of similar socioeconomic dashboards for other regions. One such dashboard was the London Datastore Dashboard~\cite{LondonDashboard} which provides constantly updated information and data on the London area. The key features of the London Dashboard is that a large variety of datasets (over 640 separate datasets) are available, and on the homepage it provides easy to understand line charts on all aspects of the region. Users can view further information by clicking on a specific data category which presents more detailed graphs.

Another dashboard used as an example was the Dublin Dashboard~\cite{DublinDashboard}. This dashboard provides a wide variety of information and distinct visualisations on the Dublin area, but the customers commented on how many clicks are required for a user to reach any actual information as the Dublin Dashboard homepage contains no information or visualisations, simply a list of links to other parts of the site. We immediately recognised a lack of immediately-accessible information as a potential pitfall of our own dashboard, and were careful to note this as an important non-functional requirement.

% What the team initially agreed compared to what the team delivered at the end

Of all the requirements the team initially agreed upon with the customers, the majority of them have been addressed and either completed as initially suggested
or an alternative has been put in place. The only large initially discussed task the team will not be completing is building our own API but it was concluded
with Eva and Tony during a meeting that this was not necessary anyway and because there were a lot of issues concerning availability of data, it would be
extremely difficult and the team could use the time the team would have spent working on that more effectively, improving other, more important parts of the site.

The customers have been consistently happy throughout, and not at any point did they ask us to remove something the team had done, except for small details such
as shadowing behind text. They made various suggestions during the meetings but even most of those were ideas that the team had already thought about but were
yet to implement.

Overall the final product fits the initial brief almost exactly and the customer is happy that what the team have produced meets their requirements. That
is, afterall, the most important thing.


% (Under Construction)

%==============================================================================
\section{Team Dynamics}
\label{sec:team-dynamics}

The assignment of members to project teams was randomised and sorted based on degree discipline, with our team being
Computing Science oriented. Our team was randomised, however two of us were close friends and another two were from the
same country which helped with initial team cohesion. One of our first tasks was to add one another on Facebook as a means of
team communication, however in addition to this the team also used Slack, a cloud-based team collaboration tool. We found this to
be of much better use than Facebook due to the organisation of discussion groups (e.g. one for general communication, one for
minutes of meetings). We werearly on in the project life cycle
but introduced the SVN component approximately half way through the process, the team were also able to integrate Jenkins and SVN into this tool to monitor the success of project builds,
and also to monitor when new commits are made to the repository. We had Jenkins integration in hindsight the team wished the team had used this from the
start. At the early stages the team also organised team roles: Lewis as Project Manager; Isaac as Technical Lead; Michael as Client Liason;
Ross as Retrospective Manager and Branko and Takis were general Software Development Engineers. In terms of agreement of these roles they were unanimous
based on the personal traits of each team member and this was consistently adhered to throughout the project.

One particular observation the team noticed early on in the process was the commitment of team members to team meetings and also
to efforts made in developing the project. The university had timetabled one all day lab session per week that was dedicated
to team project work, and on these days the team had the majority of our team members attend. The problems seemed to arise when
team members did not show up for additional team meetings arranged well in advance, sometimes this was due to mitigating
circumstances such as ill health or part-time job commitments, and whilst this was acceptable the main issue was the fact
that at this stage team members seemed to give no prior notice that they would not be available and consequently only
half the team showed up to these meetings on average. As the process continued the consistency of team members attending these
extra meetings improved albeit if they showed up later than initially arranged.

In terms of effort put into the project, there seemed to be a pattern where it was only our Tech Lead that was committing the most
work into the project in the initial stages. This was down to the fact that he has professional web development experience from previous
years while the rest of the team have limited experience. However once the site was partially functioning as a proof of concept, it
encouraged more team members to make contributions to the project.

% to be continued


%==============================================================================

\section{System Implementation}
\label{sec:system-implementation}

In software engineering there are many ways to implement a project but there are certain tools that are more advantageous over others. Some tools will allow faster iteration cycles by increasing automation, reducing boilerplate code, or faster prototyping. Other types of tools can help drive code quality improvements by providing code quality metrics such as code coverage analysis in testing, documentation coverage analysis, and warning of unused code. These tools are invaluable in modern software development and will be covered throughout the experience report. This section focuses on the specific system implementation, and the tools used throughout.

\subsection{System Architecture}
\label{sec:system-implementation:system-architecture}

For this project the team were asked to build a web application, so in order to begin development the team had to decide on a web framework that could be used as the basis of our project. We chose the Python-based Django framework~\cite{Django} as every member of the team had previous experience in developing web applications using this as part of a previous university course. There were discussions around using an alternative framework such as Java Spring and whilst the team were all proficient in the use of Java, there was a unanimous agreement that using a more familiar framework would result in faster development of a system where the requirements could change week-to-week - although not all team members were equally familiar with Django which resulted in some issues as discussed in Section \ref{sec:team-dynamics}.

Django was not the only Model-View-Controller (MVC) style framework used in the project though - the team wanted to learn new skills as well as refining existing ones. It was decided to be appropriate to introduce AngularJS~\cite{AngularWebpage} (a JavaScript MVC framework) into the project. Use of a front-end framework such as AngularJS gave us much greater control over the flow of data in our application while reducing the amount of boilerplate (repeated, menial) code in our JavaScript files. This was also a good learning experience for the team as no team members had any previous experience in using the framework.

The resulting architecture was very interesting to work with. The server-side Django code worked to supply necessary static files to clients (such as the JavaScript code files and images) as well as managing the flow of data to and from the persistent SQL database. The Django middleware worked to mediate the relationship between the AngularJS code running on client machines, and the SQL database on the server.

Many of the interactions between Django and the AngularJS code were accomplished using a technique known as ``Asynchronous JavaScript and XML'' (AJAX). This technique was popularised in 2005~\cite{ANewApproachToWebApplications, W3CAjaxProposal} as a method of creating highly interactive webpages that allow for pieces of information to be sent and retrieved from a client's browser without a full reload the webpage. In this project AJAX was heavily used to retrieve information such as HTML snippets as templates, as well as a wide variety of data in JavaScript Object Notation (JSON) format. This allowed the dashboard to be created with a very interactive feel - information could be loaded on demand when required, as well as sent for validation and storage all while being invisible to the end-user.

Several other tools were used during development of the project. These included the Bootstrap CSS framework~\cite{Bootstrap} which was used to dramatically reduce the amount of CSS code that was required to be written for the desired styling and formatting of the dashboard. D3 (Data-Driven Documents)~\cite{d3Webpage} is a JavaScript library which allows for dynamic generation of SVG elements in HTML from data. This library was used to generate all the line and bar graphs on the dashboard - the high degree of flexibility with D3 is crucial to the extensibility of the dashboard as it allows for an extremely wide variety of visualisations to be added at later stages without modification to other regions of code. Only the D3 code that renders the graphs needs to be modified.

Angular-Gridster~\cite{AngularGridster} is an open-source project which allows for extremely easy integration of Gridster-like features into an AngularJS-based project. Gridster~\cite{Gridster} is a jQuery plugin that allows for an intuitive drag-and-drop multi-column grid experience. Angular-Gridster allowed us to incorporate this functionality into the dashboard with minimal code using Angular directives (keyword shortcuts to add functionality to HTML elements). This drag-and-drop experience became a core feature of our dashboard as it allowed for a very high degree of customisation from a user point-of-view since users could resize, move, delete, and add widgets containing graphs and information to the dashboard.

\subsection{Development}
\label{sec:system-implementation:development}

The starting point of our project was to create a base template for the site: the team found a Bootstrap template that was a good starting point in terms of a colour scheme and layout that was both simplistic yet professional. Next was to decide how to implement the graphs: the team decided upon using d3 for graph visualisation as this was mentioned by a lecturer in another course the team were currently taking at university. To begin with the graphs were taken from a demonstration website with hard-coded data as more of a proof of concept that the team could implement bar and line graphs. After this the team wanted a way of making the dashboard customisable, so the team found Gridster to be very effective in terms of moving and resizing graphs. By the deadline of the second iteration this was achieved and was a great starting point to develop our site upon.

\subsection{Testing}
\label{sec:system-implementation:testing}

Another major issue the team identified was the lack of code testing. As the team were working on implementing various features into the website the team realized that the team have produced a lot of code but if it were to malfunction the team would not be certain where the problem has occurred. Having as high test code coverage as possible benefits the developers in many ways. Big changes can be made to the code and if the test cases still work that indicates that everything is still working properly. Testing code gives you further understanding of
design of the code. Now the team had to write unit tests to an already existing code which proved difficult. In order for test cases to be written one has to understand the code which is being tested. And when several people have contributed to that code it is certain that there will be pieces of code which will be inexplicable.

Good teamworking and communication helped the team reach a high percentage of test code coverage. Since the team decided to use the ``django'' framework for our website, which is based on Python the team had to look into Python junit testing. Unfortunately the documentation on ``django testing'' was rather incomplete (TODO: Modify this since django documentation is actually very complete. Perhaps due to team members infamiliarity with Python unit testing in general?). We had to carry out an extensive research on how to use ``python junit'' to test our code. There were not many sources covering that subject and the team had to use bits and pieces from different sources. By the third iteration the team had managed to raise the test code coverage significantly and started working on tests while writing the actual code.

%==============================================================================
\section{Core Practices}
\label {sec:core-practices}

Throughout the development of the project the team were also following a course named Professional Software Development.
The course's aim was to introduce the students to modern software development methods and techniques for building and
maintaining large systems and apply those techniques in a project. One of the methods that were taught
was the agile practice. Agile software development ``is a set of principles for software development in
which requirements and solutions evolve through collaboration between self-organizing, cross-functional teams.
It promotes adaptive planning, evolutionary development, early delivery, and continuous improvement, and it encourages
rapid and flexible response to change''.

The agile method that was mostly used was Extreme Programming (XP). The benefits of XP are beneficial since it focuses on
customer satisfaction, enables the continuous elicitation of requirements from the customers and insists on teamwork
both between the team members and the customers. Also, XP improves the communication, simplicity, feedback, respect and
courage of the team. It focuses on a number of simple rules that can be easily adhered to.

\begin{wrapfigure}{l}{0.5\textwidth}
\includegraphics[width=1.1\linewidth]{figures/Extreme_Programming}
\caption{Time span of tasks in Extreme Programming.}
\label{fig:extreme-programming}
\end{wrapfigure}

Following the guidelines of XP the team conducted the project with these principles in mind. Since the project was set to finish at a specific date
the team set a number of task deadlines which were due at the end of each iteration. In total five iterations
(including the final demonstration) have been set by the university. The first iteration was our initial meeting with
the customers. From this point the team have understood the concept of the project, gathered initial functional and non-functional
requirements of the project and also proposed some ideas to our clients. At the end of the meeting the requirements were
translated into user stories, from where a number of tasks have been created to provide possible solutions to the user stories.
By the end of this iteration a meeting with the customers was arranged in order to demonstrate the current structure of the
project, asking it fits their needs and collect new requirements for the following iterations. This process has been followed
until the end of the project including the final demonstration. It is also fair to say that in between each of the iterations the team
aimed to discover defects which in turn were translated into new user stories and fixed during their respective iteration,
if not the following.

The first step the team took was to keep good relationship with their customers. During the meetings and iterations, the relationship
between the customer and the team was very close, friendly and completely understanding. The two parties kept in touch at least once or twice during the
iterations and the team have always asked for their customers opinion whenever something had to be done that could affect the agreed requirements.
The customers were not just sources of information and guidance, instead they were an active member in the team and have very
efficiently helped in the development of the project.

During every iteration the team followed a planning method. Since the first meeting has been concluded there have been generated a series of
user stories. Those user stories have been carefully examined and graded by the team and have been assigned into specific milestones, i.e.
end dates, based on their grade. In addition to that, the clients were informed about these dates and they were informed of what would be finished in the next iteration. Although the user stories were graded and set to specific milestone, they were not fixed, instead they were dynamically changed based on the work done so far, random implications that have risen and the addition of new user stories. Furthermore, every user story had its respective owner/s which was/were resposible for the research and development of the story. In relation to the planning methods , it is good to mention that every requirment's milestone set, was after research of the estimated time that can take including external factors such as sickness/assignments/travelling abroad.

%Leaving user stories aside, the team had a specific plan during the iterations. At the end of every iteration the team always assembled
%and participated in a retrospective. In the retrospective the team followed  the four L's scheme , where every team member was given a set
%of sticky notes with the titles of either Liked, Learned, Lacked and Longed for, on top of it . Then every team member would fill the sticky notes with their respective personal opinions and statements which were later sticked to a whiteboard and pictured. The whole process was
%saved and presented to our tracking system Trac . After the end of every iteration and therefore retrospective our team would start the new iteration the following week with a brand new plan meeting where we would discuss the previous iteration, make a plan of the new iteration
%assign the new user stories to the appropriate members and get a heads up of what needs to be done for the following weeks.

%==============================================================================
\section{Process Improvements}
\label{sec:process-improvements}

% We talk about the improvements the team has made, such as improvements make to ticket management.

As part of the project, the university wanted the team to use three tools throughout the process: Subversion (SVN) for our version control
repository, Trac as our ticket management system, and Jenkins for project builds. Before the start of the project the team had been
shown using tutorials how to set these up which was helpful for the long run. We found SVN relatively straightforward to use
in terms of making commits to the repository, however the team would have preferred using Git as the team have all
had previous experience using this and in general seems much more of a user-friendly experience. Trac was very helpful in
terms of monitoring existing tasks that required work upon, and also the use of Wiki pages allowed for better team organisation
such as project information and summarising retrospectives. To begin with Jenkins was mostly used for ensuring that builds of
the project after commits were successful, however at this point the team did not have much testing on our site. We also had an issue
whereby each time a commit was made to the repository, it would take just over three minutes to build each time as the virtual
environment had to be re created every single time. Thankfully the team were able to reduce the build time by changing the point at
which the virtual environment is created in the build script. Further into the project the team also integrated Cobertura for code
coverage, and this was extremely helpful in terms of the depth of existing code coverage and what areas were still to be tested.

As part of agile development, the team undertook a retrospective at the end of each iteration. The goals of retrospectives are
to establish where the team has done well and identify process improvements for future iterations. A retrospective is started
by asking each team member to write their thoughts on how they've felt about the previous iterationon post-it notes.
The team divided these into four categories: Liked (things the team liked); Learned (things the team have learned);
Lacked (things the team have been doing but could be done better) and Longed For (something that's desired or wished for).
Once each team member has written their thoughts down, the Retrospective Manager(Ross) asks each team member in turn
what they said for each category by placing their post-it note on a wall or board in order to collate them. By doing this,
the team is able to find commmon opinions whilst also finding thoughts that other team members previously did not
consider. To conclude a retrospective, the team summarises the points made i.e. how much have we progressed as a team
and where are the problem areas. From this, the team is able to establish new goals and objectives for the next iteration
which may involve creating new tickets on Trac. At the next team meeting, the team clarifies what tasks need done over
the next few weeks before the next customer meeting.

Retrospectives were great for the team in terms of finding process improvements. In the first iteration it was more to
do with indivdual efforts such as showing up to team meetings on time, more commits to the repository and clarification
of who is working on which tickets. Later iterations had more productive changes due such as more test cases, code
documentation and better ticket management. The most useful improvement the team has made in the scope of the project
is referencing tickets in commits to the repository: whilst this happened later on in the process, it enabled team members
to identify clearly who is working on which tasks and the progress made towards it.

%------------------------------------------------------------------------------

\section{Continuous Integration}
\label{sec:continuous-integration}

Continuous Integration (CI) is a development practice that requires software developers to integrate code into a shared
repository (e.g. using GIT or subversion) several times a day. Each check-in is then verified by an automated build,
allowing teams to detect problems early.By integrating regularly, you can detect errors quickly,locate them more easily
and solve the problems faster. CI has been adopted as part of extreme Programming (XP).Since the team has adopted the XP
methods it was only natural for the team to follow the CI practices.

The first practice of the CI is about using a shared code repository. As stated in the previous sections the team has
used subversion (a revision control system) that helped the team to share and manage their code very easily. Every team
member could update to the latest version of the project, then work on their own versions of code changing it as
necessary and then were able to commit their changes in order for everyone to be up to date.

The second practice of CI rests upon automating the build, have it build fast, be self tested and automate its
deployment. The team has used anautomation server called Jenkins and a test server to host the web application of the
project. By using Jenkins it has resolved the issues of automating the build process (with the help of a python build
population script). It has also provided a way of self testing, since as the project was growing tests have been
implemented to test our code, and Jenkins made sure the tests passed before building and updating the live test site.
When Jenkins finishes the build and its testing it deployed to a live test server so everyone (both the team members and
the clients) could see and evaluate the results.

Furthermore , CI specifies that the developers of the project should commit at least once a day and that commit should
be built as soon as is commited. That is a practice that was not followed. Since the team was composed by students
such a practice was difficult to follow. As a professional developer it is fairly easy to commit every single day
since you have the time required to do so. In the case of a student, even if he split his working time to every single
day, the job done would be minimal or even incomplete,causing the the servers to fail. Also, because of the nature of
our project a big amount of time was spent on researching for methods or libraries to counter some problems, and that is
not a commit material. For these reasons, the commits were inconsistent and every commit would contain a lot of new
code from its previous version, which sometimes resulted into merge conflicts which were very difficult to resolve. In
order to counter or limit the frequency of conflicts the team managers has suggested the team to be meeting twice a week
and also work on the weekends while everyone would inform on what they were working and their commit messages would tell
what and which files were changed. The ideas were indeed beneficial and have reduced substantially the counters of
conflicts and defects.
% Feel free to cut sections from my work into here


%------------------------------------------------------------------------------

\section{Project Planning and Estimation}
\label{sec:planning}

% same here

Why we had to drop certain features, how we dealt with ``feature creep'' (the introduction of previously unidentified features), and how we dealt with work estimations.


%------------------------------------------------------------------------------

\section{Data Acquisition}
\label{sec:data-acquisition}

% same here

The next issue the team had to deal with is how the team can use real data in these graphs. We started by researching the various websites
suggested by our customer such as NOMIS Scotland~\cite{NOMISScot}. We found that most of these sites were difficult to use in terms of extracting
exactly the kind of data that the team needed. Eventually the team came across Statistics.Gov.Scot Beta~\cite{StatisticsScotBeta} which was a site being set up on behalf of
the Scottish Government to be used as a centralised access point to information. In this, the team were able to find data on
hundreds of economic indicators in CSV format. We created a script that would process these files, extract only the data collected
on Dumfries and Galloway and Scotland and format this to be rendered by the d3 code that visualises the data before storing it on
our site's database. This data extracting script was implemented into the project by creating a simple population script which used the data extraction method to provide the project with the real data and in parallel to that created fake dummy data for testing purposes (I need help rephrasing this sentence).This was a major step in our project in terms of using real data, however our main issue here was that the
data was static, and at the time of download was outdated. In addition to this the files all used inconsistent date formats (e.g.
one would be quarterly, another would be monthly, another yearly) which presented a major issue in terms of a key requirement
being that users want to access the latest available data as well as looking back on previous years. We raised this issue
with our customers and they were aware of this issue beforehand, this was a relief. Later on in the development process the team decided to make changes to the data extracting script which was initially created. At first the script was only capable of extracting data from CSV formatted files which strictly modified to fit the same structure. Although this was working at first the team decided to improve the script to be able to extract data from all CSV files regardless of the inner format and structure. The team decided to keep the old converter script for demonstration purposes while the new one was being developed. The idea behind the new script was to allow administrators to add more CSV files as data sources through a relatively simple process. This process is thoroughly described in the admin panel with examples for quick and easy understanding.
Once the new script was fully developed and tested the team implemented it fully into the project thus removing the obsolete old version.


%------------------------------------------------------------------------------

\section{Customer Management}
\label{sec:customer-management}

%We talk about how the team appointed a customer liason to communicate with and receive feedback from our customer, how the team dealt with
%demonstrations, how the team dealt with change of expectations and changes to customer requirements. We also should talk about the
%introduction of requirements from other teams in terms of integration into the Crichton site.

One of, if not, the most important aspect of a successful software devlopment process is the interaction and relationship
between the customer and the developers. The best way to ensure this is possible is by ensuring good communications
both in person and otherwise, in this instance the team appointed a Customer Liason (Michael) to be the link between the
team and the customer. Ideally we would have liked to meet the customer more but given the distance they had to travel
for each customer meeting from Dumfries and Galloway, this was not feasible, so the team regularly emailed the customer
to make them aware of any queries or issues relating to the project.

The team's face-to-face interaction with the customer took place in five meetings: the first was to meet the customer
for the first time and gather requirements, the following meetings were to demonstrate the progress made in each
iteration. From requirements gathering, the team were elaborated on the project description in order to create user stories,
and agreed that over the 7 month period it would be feasible to implement most (if not all) the desired features into the
dashboard. One of the team's initial challenges was how to convey our requirements over to the customer who was not particularly
technologically literate, but the team did well to come to a mutual understanding over what was expected from both parties.
Over the next few meetings, the customer seemed very pelased by the progress made from each iteration. What was extrememly
pleasing was how the team identified possible extra features that the customer never initally though about whilst at the
same time the customer pointed out areas of our project the team had not previously considered. Generally, meetings
were conducted in stages: the first involved the team demonstrating the progress made using the live deployment of the
dashboard (this was also useful for them to be able tocheck our progress outside of meetings and also for their
stakeholders to see the progress the team has made); clarify any issues the team was currently facing (i.e. a feature
that may not be an ideal solution or is not feasible to implement within the time frame); agree on what is expected
for the next iteration/meeting and finally ask the customer for any further feedback/questions they may have.

An issue that arose when interacting with the customer was that despite previously agreeing to implement data access
via an API, the team realised half way through the process this may not be feasible due to inconsistent and outdated
data; and the lack of helpful documentation provided by relevant government sources. Thankfully the customer recognised
this and got the team in touch with the owner of the website we had previously acquired CSV files from for the graphs on
the dashboard. Whilst this was positive to begin with (we were told data would be updated within 24 hours), this was
never carried out by the site owner and thus could not be used for our project. The site's developers did not offer much
assistance either and did not get in touch with the team until the customer intervened. Thankfully the team agreed with
the customer that using the CSV importer was a practical alternative given the fact that most exisitng dashboards use
CSV files as well, as opposed to dynamically through an API.

A key aspect of the project was the fact that the team not only had to communicate with the customer but also with
other project teams. The team was contacted by other teams regarding the possibility of integrating their projects into
our dashboard, this was initally agreed however we agreed with most teams that each project would be as standalone as
possible. The team were in regular contact with the project team involved in moving the Crichton's website over to a
new platform who would be able to integrate the dashboard into their homepage.


%------------------------------------------------------------------------------
\section{Conclusions}
\label{sec:conclusions}

Explain the wider lessons that you learned about software engineering,
based on the specific issues discussed in previous sections.  Reflect
on the extent to which these lessons could be generalised to other
types of software project.  Relate the wider lessons to others
reported in case studies in the software engineering literature.

% Some possible lessons: software development is an iterative process, a project is never fully finished, requirements
% will have to be reconsidered, it takes time for a team to be fully cohesive

%==============================================================================
\bibliographystyle{plain}
\bibliography{dissertation}
\end{document}
